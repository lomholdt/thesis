\subsection{Competitor overview}\label{sec:competitor-overview}
To get an overview of the different systems from a competitive standpoint, we introduce an objective analysis based on a list of features consisting of common functionalities and our own additions, all based on our initial research within the field of CRSs. The features are based upon the common features and functionalities that are seen in the different systems. The list of response systems is based on our own research. The systems mentioned here covers a great deal of the response systems that are available today and most of them are commercially available.

\subsubsection*{Features}
The following list of features will be used to compare the existing solutions.

\begin{itemize}
    \setlength\itemsep{0em}
    \item Web/Software/Hardware based platform
    \item Multiple question types
    \item Multiline question and answers
    \item Math notation
    \item Source code notation
    \item Supports image upload as questions
    \item Timed questions/auto closing questions
    \item Payment model
\end{itemize}

The first feature determines if the system is web, software or hardware based. Some systems are only accessible online, while others are purely hardware based, and users needs to purchase specific hardware to use it. A few systems are also purely software based and they are created for use on computers/tablets or phones and does not have any specific hardware.

Many CRSs offers multiple types of questions. An example is one of the more common ones,  the multiple choice question where users can select an answer, where one (or more) of those answers are correct. Another example could be answering on a scale, an open ended question and so on.

Some CRSs offers the possibility to ask questions with multiple lines of text. This enables teachers to ask questions spanning several lines, useful in certain scenarios. The same principle is applicable to answers. This is especially important if the representation of the question is important to ease learning. For example, it's easier to read source code that is indented and formatted nicely instead of being cramped together on one line.

Math notation is used for asking questions and choosing answers with mathematical expressions. This means that the system supports the ability to show formulas formatted like this: $$x = {-b \pm \sqrt{b^2-4ac} \over 2a}$$ instead of showing pure inline text. 

Source code notation is defined as a system that supports (at a minimum) the ability of multiline questions and answers, and uses a monospaced font for it. It could also include syntax highlighting, but is not required. 

Image upload is the possibility to upload an image as a question. This enables the users to use any format for a question or answer, since they can simply upload an image that looks exactly the way they want it to.

Timed questions enables the users to auto close questions after a certain amount of time. In some cases the remaining time available to answer the question is visible to the responders.

The payment model differ from system to system. We will not go into details about pricing, but they have been included to give a greater context, and as part of their competitive advantages.

All the following CRSs will be analysed with these features taken into consideration.

\subsection*{Kahoot}
Kahoot is a web based CRS, that focuses on teaching children. Kahoot themselves states that  \emph{"it gives students a voice in the classroom, and allows educators to engage and focus their students through play and creativity"}\footnote{Retrieved on 2016-3-25 \url{https://getkahoot.com/support/faq/\#is-kahoot-a-social-media-tool}}. Kahoot is purely web based and supports multiple question types such as quizzes, discussions and surveys. It has image upload support to make questions more engaging, but it does not support multiline questions and answer, nor does it have support for math notation and source code. It does have timed questions though. Even though Kahoot works well, it is only created with simple questions in mind, that would fit in one line, also their colorful design seems to appeal to a younger audience. Kahoot is free to use and has no payment options.

\subsection*{Socrative}
Socrative is a web based platform that supports multiple question types. It has support for multiline question and answers, but not for monospaced fonts, thus math notation and source code notation is not directly supported. There is support for imageupload, so it's possible to see an image as a question. The system is a so called \emph{K-12} solution, which means it is build for elemanty to high schoolers. Socrative is free to use, but does reserve the right to introduce payed extra features.

\subsection*{Poll Everywhere}
Poll Everywhere is a response system with several different possibilities for voting. You can \emph{"(...) respond via the poll's web page on PollEverywhere.com, or via an embeddable voting widget, or on a mobile web browser using PollEv.com, or even through Twitter"} beside sending a text to a specified number\footnote{Retrieved on 2016-3-25 \url{https://www.polleverywhere.com/faq\#how-to-vote}}. Poll Everywhere will primary be web based since questions are created on their web page. The system support different kind of questions such as multiple choice, open ended questions, clickable images and it has a Q\&A/brainstorming feature. Furthermore, Poll Everywhere has the possibility of displaying short single lines of Latex and thus displaying math. A blog post on their site from 2012 contains a video which shows how to insert Latex at the very end of the video\footnote{Retrieved on 2016-3-25 \url{http://www.polleverywhere.com/blog/its-what-you-wanted-image-support-math-equati/}}. It has been impossible to find the information anywhere else.

Poll Everywhere doesn't support multiple lines in their questions and answers, and nor does it support source code in the questions/answers. It does support images as answer but not as a question. It's not possible to time questions on Poll Everywhere and it's only possible to have one active question at any time. 

\subsection*{iClicker}
iClicker started out as a purely hardware based CRS, but has recently updated their solution with a web based platform for answering as well (including apps for iOS and Android devices). This means that teachers still need a piece of software, but students can use a web based platform for responding. It supports multiple question types. The iClicker software is capable of sending a screendump from the teachers computer to the web application (or app), thus enabling the teacher to use any software of their choosing for asking questions. In principle this enables multiline questions and math/source code notation but it's not natively supported. 

\subsection*{Mentimeter}
Mentimeter is a purely web based platform and also one of the more feature rich. It supports seven different question types, but does not have any multiline support. It does however offer a description field for each question, but it is limited to 100 characters making more complex question asking hard to do. Even though Mentimeter is very feature rich, it does not have support for math notation, nor does it have source code notation.  Image upload is not supported, but timed questions are in 2 varieties. Mentimeters payment model is subscription based, with several types of subscriptions. 

\subsection*{Tophat}
Tophat is similar to Mentimeter and Poll Everywhere in form and function. It's a web based solution capable of taking attendance and asking questions. Tophat features 6 different question types, a discussion function and a tournament mode which let students compete against each other. Tophat supports displaying math and code by including tags like [math]\emph{some equation}[/math] and [code]void main()[/code]. Code is simply displayed in a grey box with black text and is not highlighting any syntax. This also implies that it's possible to ask multiline questions. However, it's not possible to create multiline answers nor write code in an answer. It's possible to set a timer on questions in order to change the status from active (labeled as "Homework") or review to review and closed. Tophat features a few more statuses but these cannot be used when scheduling a question. In order to use Tophat, every student must pay a fee corresponding to how long they intend to use the system.

\subsection*{Learning Catalytics}
Learning Catalytics is a solution made by Pearson. The system seems feature rich from instruction videos and guides, but it's not possible to test for ourselves since it's behind a pay wall. Learning Catalytics seems to be software and web based from videos found at their website\footnote{Retrieved on 2016-03-28 \url{https://learningcatalytics.com/pages/training}}. The videos also reveal an advanced grading system based on how many tries it took to answer the correct answer. Learning Catalytics features options for the instructor to make a seat map which is supposedly used to tell students who to team up with when solving group exercises. An "I don't understand this"-button can be pressed to tell the instructor that you don't understand the material taught or the question asked. In order to use Learning Catalytics, students or their institution must pay for a subscription while instructors seemingly can use it for free but. Still it's unavailable for us to test.

\subsection*{Informa}
As previous mentioned, Informa is a system and tool for teaching Java. It's build and tested at University of Lugano. Informa is a traditional response system while it supports teaching Java with different approaches and techniques. For example: It's possible to ask students to highlight specific parts of some provided code or determine types of already highlighted code \cite[p.~2]{Hauswirth09}. Furthermore the system supports a method for asking students to draw diagrams and flowcharts \cite[p.~3]{Hauswirth09}. 
The system consists of two different clients, an instructor client and several student clients. 

\subsection*{Renaissance}
Renaissance is a hardware and software based system where users respond using the \emph{Renaissance Responder}. It is a simple multiple choice system, so multiple question types are not directly supported. Answers are limited to A, B, C, D and E and True or False via the responder hardware, multiline questions and answers are therefore not supported. Math notation is not possible but the system does have a build in calculator. Source code notation and image upload is not supported. It is unclear if timed questions are supported, as the documentation for the system is sparse, but based on the other features of the systems it is not likely. The payment model for the system is based on hardware purchases.

\subsection*{Classroom Learning Partner}
The Classroom Learning Partner (CLP) is a software based system where \emph{"the goal of the [...] project is to increase instructor-student interaction and student learning in large classes by developing software to support the use of in-class exercises"}\footnote{Retrieved on 2016-03-29 \url{http://publications.csail.mit.edu/abstracts/abstracts07/kkoile/kkoile.html}}. The system is different from the other mentioned ones, in terms of input type, where CLP relies heavily on written input, the so called \emph{Digital Ink} input, or simply hand written input \cite{koile2007supporting}. The system supports multiple question types and also multiline questions. It is unclear wether the system supports mathematical notation, but it does support source code, since the text font is a monospaced font. It's possible to upload images, and even draw the questions. Timed questions are not available.



\subsection*{Shakespeak}
Shakespeak is the type of response system that uses traditional software in order to work. Unfortunately it will only run on Windows. Shakespeak integrates directly into Microsoft PowerPoint and adds an extra menu. From here it's possible to ask questions to an audience. In order to vote or answer a question created with Shakespeak, the responder can text, answer online or send a tweet with Twitter\footnote{Retrieved on 2016-04-01 \url{https://www.shakespeak.com/how-to-use-shakespeak-during-all-your-presentations/}}. It supports two kinds of questions: multiple choice and open ended questions. Since the response system lives directly in Microsoft PowerPoint, it will potentially be possible to ask questions with math, source code and images but it's not supported as such. It's possible to use Shakespeak for free with a limitation on the amount of users who are allowed to answer questions.



\subsection{Summary}
Based on the above, we have summarized the features in table \ref{tab:overview}. Related to our research question, we have added coloured highlights to the columns that underline the fact that the system is \emph{enhanced for IT educational purposes}. As it can be seen in table \ref{tab:overview}, most systems does not have support for math and source code notation at the same time. Only Tophat supports both features based on our definition, and does well with mathematical notation, but less well on code. In fact, it only supports code for questions and not for answers. Even though this is not part of our definition for code support, it is one of the features that we find relevant to expand on.
None of the systems have syntax highlighting, and most only have oneline options for answers, this encourages us to include such features, since many people in the software industry seems to have syntax highlighting in their code editor. 

% We wish to investigate whether a Classroom Response System that is enhanced for  IT  educational  purposes,  is  able  to  enhance  classroom  interactivity  and thereby  increase  the  learning  outcome. We wish  to  evaluate  the  system  by gathering empirical data by conducting a pre and post survey on students using our system.


% Tophat er cool til matte syntax men kode kan kun vises i spørgsmål og ikke i svar!


\begin{landscape}
\thispagestyle{empty}
    \begin{center}
        \begin{table}[H]
            \begin{tabularx}{\paperwidth}{ |X|X|X|X|X|X|X|X|X| } 
             \hline
                 & Web \newline Software \newline Hardware based & Multiple question types & Multiline support & Mathematical notation & Source code notation & Supports image upload as questions & Timed questions/auto closing questions & Payment model \\ \hline
                 
              Kahoot                & Web   & No    & No    & \cellcolor{red!25}No    & \cellcolor{red!25}No    & Yes   & Yes   & Free \\ \hline
              Socrative             & Web   & Yes   & Yes   & \cellcolor{red!25}No    & \cellcolor{red!25}No    & Yes   & No    & Free \\ \hline
              Poll Everywhere       & Web   & Yes   & No    & \cellcolor{green!25}Yes   & \cellcolor{red!25}No    & Yes   & Yes   & Subscription \\ \hline
              iClicker              & All   & Yes   & No    & \cellcolor{red!25}No    & \cellcolor{red!25}No    & Yes   & No    & Mixed based on solution \\ \hline
              Mentimeter            & Web   & Yes & No   & \cellcolor{red!25}No    & \cellcolor{red!25}No    & No    & Yes   & Subscription \\ \hline
              Tophat                & Web   & Yes & Questions only & \cellcolor{green!25}Yes & \cellcolor{red!25}No, only in questions   & No    & Yes & Subscription \\ \hline
              Learning Catalytics   & Web \newline Software   & Yes    & Yes  & N/A   & N/A   & Yes   & N/A   & Subscription \\ \hline
              Informa        & Software & Yes & Yes  & \cellcolor{red!25}No & \cellcolor{green!25}Yes   & No    & No    & Research project, free \\ \hline
              Renaissance           & Software \newline Hardware & Yes & No & \cellcolor{red!25} No & \cellcolor{red!25}No & N/A & No & Hardware purchase \\ \hline
              Classroom Learning Partner & Software \newline Hardware & Yes & Yes & \cellcolor{red!25}No & \cellcolor{green!25}Yes & Yes & No & Research project, not available \\ \hline
              Shakespeak            & Software \newline Web & Yes & Yes & \cellcolor{red!25}No & \cellcolor{red!25}No & No & No & Subscription \\
              \hline
              %\Xhline{2\arrayrulewidth}
              %CRSFIT                & Web & No & Yes & \cellcolor{green!25}Yes & \cellcolor{green!25}Yes & No & No & N/A \\ \hline
            \end{tabularx}
            \caption{Overview of Classroom Response Systems}\label{tab:overview}
        \end{table}
    \end{center}
\end{landscape}





