\section{Conclusion}

%We wish to investigate whether a Classroom Response System that is enhanced for IT educational purposes, is able to enhance classroom interactivity and thereby increase the learning outcome. We wish to evaluate the system by gathering empirical data by conducting a pre and post survey on students using our system.


% Response systems har eksiseter i mange år, i froskellige former og vi har besrkevet nogle af de nyere både fysiske og software og web, 
Response systems has been around for many years and they come in various shapes. Recent versions are however made primarily for web and mobile devices or has moved from being physical devices to web and software based solutions.

%så har vi forsøgt at danne overblik over hvordan vores passer ind
%by presenting the code in different systems and us
Based on our own experience, we have observed that asking questions involving code has been done using pure (unformatted) text in a CRS or by presenting questions on slideshows and gathering answers via a CRS. Furthermore we found that most research done within this field, is focused on physical devices instead of the web based ones. Also the ability to ask technical questions with code and math formatted syntax is not present or is not working properly.

In order to gain in depth insights into the field of classroom response systems, we have explored many of the existing solutions and made a competitive analysis of the market. We found that indeed, most systems does not support technical input such as well formatted code and mathematical notation. We also found and highlighted areas in the competitive analysis which we have to take into consideration if we were to enter the market of CRSs.

% for så at udvikle det
We started developing our own response system, CRSFIT, that supported the missing features, in order to test if it was able to enhance classroom interactivity and increase the learning outcome. We created and conducted a test inspired by \citeA{siau2006use} to compare two courses, one using a CRS without the code and mathematical notation features and one using CRSFIT. 
The results gathered from these tests proved to be inadequate to show if student had increased their learning outcome, and further research must be made to gain a more conclusive answer. Students using CRSFIT and Mentimeter do however report that using a response system increases their interaction (figure \ref{fig:perceived_ease_of_use_and_usefulness}).






% gennem test af systemtet (framework for at teste) -> 2 forskellige klasser med to forskellige sysytemer for at se om vi kan følge med o
% vores resultater er ikke konklusive og der er brug for enlængere varende undersøgelse for at se om vi reelt kan løfte folks participation in class 
% som også fremgår af diskussionen
% we weere not able to enhance classroom interactivity, so we are not able to say anything about increased learning outcome
%A result of this project is also a framework for testing how our system could improve students interactivity.




