%\subsection{Introduction}
Based on the above, we will 

\subsection{Competitive analysis}



%% Tror vi skal være enige om rækkefølgen, for vi skal vide hvad vi har lavet og ikke har lavet nu :| Om systemet eksisterer eller ej, og om vi har ellle ikke har undersøgt markedet der og alt det lol

%Arh... Ja. Går ud fra vi på det her tidspunkt har undersøgt markedet, konkurrenter osv. Jeg tænker mig lige om. 

%Jeg tror vi skal prøve at tænke:
%Vi ser et problem -> vi undersøger hvilke løsninger der er -> ingen (som gør det vi tænker er løsningen) -> vi laver en løsning -> vi undersøger om den virker




%Vi ser et problem ->
%Introduction
% Lit Review
% Method - hvordan går vi til det her problem?

%vi undersøger hvilke løsninger der er/markedet/konkurrence osv -> konklusion på det: Der er ingen der fylder hullet ud
% 'Competitor overview' - kunne ikke komme på et bedre navn - måske Existing solutions
% Competitive Analysis
% 

% threat of entry
Considering the complexity of most CRS'es, fortunately the economies of scale are relatively transparent. For a web application like this, the scaling can come, as more users start using it. For example, the hosting of the system could be done with as little as 3 servers, at a very low rate. If more capacity is needed, scaling horizontally could be as easy as the click of a button. The cost would of course then grow, but only when the user-base does as well.
The demand-side benefits of scale is, as explained in section \ref{sec:threat-of-new-entrants}, likely to be influenced by user needs. Many of the mentioned CRS are used \emph{'on the spot'}, and rarely hold features, that are critical for future use (such as grading). This is also backed by the fact, that the customer switching cost is so low on the web based systems. On the other hand, this also means, the possibility of users leaving our system without much consideration is relatively high, so gaining customer loyalty should be taken seriously.
Even though one could invest hours on end, designing and implementing and maintaining a CRS, it is indeed possible to build one without large financial resources. Of course, complexity can grow, depending on the system features, but taking the CRS definition into consideration, a \emph{plain} system adhering to this definition, should be very doable in a reasonable amount of time, without large financial support, even at a competitive level.

As mentioned, it is also a noticeable advantage to be first, and well established on a market. If the market is already saturated, entrants will have a really hard time entering the market, unless they have something extraordinary to offer. It would seem that many have tried (referring to the many systems mentioned here), and many also appears well established, but none seem to be dominant, and all offer a different set of features. This makes the idea of gaining tracking with niche features as a plausible strategy.

% power of suppliers
Even though this force is extremely relevant for the strategic planning of an entrant, the fact that our CRS is fully web based software, the supplier group that we are concerned about is less relevant. In our case we are primarily interested in hosting and data-storage, and the market for this is very versatile. One consideration might be that once a supplier has been chosen, the switching cost is high, due to the fact that it would be necessary to move all the data. Depending on the company's data history policy this could become the largest consideration in terms of bargaining power of suppliers, since the hosting providers could charge more, without overpricing the switching cost.


% power of buyers
The bargaining power of buyers is, however, more relevant when considering threads to ones position in the market of response systems. Since there's several classroom response systems on the market, a potential buyer may consider a lot of different systems before deciding on a system to use. The buyers of any CRS may consider their potential profits or any gain they will get before investing and implementing such a system into their courses and teaching. Since we aim at increasing interactivity in classrooms which teach IT courses, a potential buyer may ask what they will gain from using our system. When the buyer \emph{"... earns low profits, which creates great incentive to lower its purchasing costs."} \cite[p.~141]{porter1979competitive} price may be an important factor, because it can be hard to tell what the buyer will actually gain. Considering that our system is free of charge, we may be having a strong selling point before comparing features with direct competitors. Other free alternatives such as Kahoot and Socrative (see Table \ref{tab:overview}) offer similar features as we do, but they do however lack the ability to display code and mathematical expressions.


% threat of substitutes
It is necessary to take similar solutions and alternatives into account when considering threats of substitute products or services. At ITU there's a build in feature in LearnIT\footnote{A system based on Moodle which is an open-source learning platform. See \url{https://moodle.org}} which allows teachers to ask questions to students. The thread of substitution may be high if this feature's \emph{price-performance trade-off} is found acceptable by teachers and students. However, this feature is not as feature-rich as most CRS's and it may not fully replace any CRS. Since this feature is already present (at ITU) and paid for, the price-performance trade-off could be found to be acceptable and thus eliminate the need for an actual CRS. Other similar learning platforms may facilitate simple question features as well. It will therefore be necessary for us to offer more value than what a possibly already available system can provide, in order to be considered a better alternative or substitution. 


% Rivalry
As we have mentioned in the previous sections, there are quite a few CRS'es on the market. They all vary in quality, features and pricing. This is an important aspect to consider when trying to gain market share. We have devoted section \ref{sec:competitor-overview} to show a list of relevant competitors and their features and pricing structures.








