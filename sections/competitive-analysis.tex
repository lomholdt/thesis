\section{Competitive Analysis}
To get an overview of the different systems from a competitive standpoint, we introduce an objective analysis based on a list of heuristics that we find relevant based on our initial research within the field of CRS. The heuristics are based upon the common features and functionalities that are seen in the different systems. The list of CRS are based on our own field research. A substantial amount of time has been spent on finding these systems, and the ones mentioned here covers a great deal of the CRS'es that are available today.

\subsection{Heuristics}

The following is a list of the chosen heuristics.

\begin{itemize}
    \item Web/Software/Hardware based
    \item Multiple question types
    \item Multiline question and answers
    \item Math notation
    \item Source code notation
    \item Supports image upload as questions
    \item Timed questions/auto closing questions
    \item Payment model
\end{itemize}

The first heuristic determines if the system is web, software or hardware based. Some systems are only accessible online, while others are purely hardware based, and users needs to purchase specific hardware for it to work. A few systems are also purely software based, and is created for use on computers, and does not have any specific hardware.

Many CRS offers multiple types of questions. An example is one of the more common ones, the quiz, also known as the multiple choice question where users can select an answer, and one (or more) of those answers are correct.

Some CRS offers the possibility to ask questions with multiple lines of text. This enables teachers to ask questions spanning several lines, useful in certain scenarios. The same principle is applicable to answers.

Math notation is used for asking questions and choosing answers with mathematical expressions. This means that the system supports the ability to show formulas formatted like this: $$x = {-b \pm \sqrt{b^2-4ac} \over 2a}$$

Source code support is defined as a system that supports (at a minimum) the ability of multiline questions and answers, and uses a monospaced font for it. 

Image upload is the possibility to upload an image as a question. This enables the users to use any format for a question, since they can simply upload an image that looks exactly the way they want it to.

Timed questions enables the users to auto close questions after a certain amount of time. In some cases the remaining time available to answer the question is visible to the responders.

The payment model differ from system to system. We will not go into details about pricing, but 


All the following CRS will be analysed with these heuristics taken into consideration.

\subsection{Kahoot}
Kahoot is a web based CRS, that focuses on teaching children. Kahoot themselves states that  \emph{"it gives students a voice in the classroom, and allows educators to engage and focus their students through play and creativity"} \footnote{Retrieved on 2016-3-25 \url{https://getkahoot.com/support/faq/\#is-kahoot-a-social-media-tool}}. Kahoot is purely web based and supports multiple question types such as quizzes, discussions and surveys. It has image upload support to make questions more engaging, but it does not support multiline questions and answer, nor does it have support for math notation and source code highlighting. It does have timed questions though. Even though Kahoot works well, it is only created with simple questions in mind, that would fit in one line, also the colorful design is appealing for an audience of younger age.


\subsection{Socrative}
Socrative is a web based platform that supports multiple question types. It has support for multiline question and answers, but not for monospaced fonts, thus math notation and source code notation is not directly supported. There is support for imageupload, so it's possible to see an image as a question. The system is a so called \emph{K-12} solution, which means it is build for elemanty to high schoolers.


\subsection{Poll Everywhere}
Poll Everywhere is a response system with several different possibilities for voting. You can \emph{"(...) respond via the poll's web page on PollEverywhere.com, or via an embeddable voting widget, or on a mobile web browser using PollEv.com, or even through Twitter"} beside sending a text to a specified number\footnote{Retrieved on 2016-3-25 \url{https://www.polleverywhere.com/faq\#how-to-vote}}. Poll Everywhere will primary be web based since questions are created on their web page. The system support different kind of questions such as multiple choice, open ended questions, clickable images and it has a Q\&A/brainstorming feature. Furthermore, Poll Everywhere has the possibility of displaying short single lines of Latex and thus displaying math. A blog post on their site from 2012 contains a video which shows how to insert Latex at the very end of the video\footnote{Retrieved on 2016-3-25 \url{http://www.polleverywhere.com/blog/its-what-you-wanted-image-support-math-equati/}}. It has been impossible to find the information anywhere else.

Poll Everywhere doesn't support multiple lines in their questions and answers, and nor does it support source code in the questions/answers. It does support images as answer but not as a question. It's not possible to time questions on Poll Everywhere and it's only possible to have one active question at any time. 


\subsection{iClicker}
iClicker started out as a purely hardware based CRS, but has recently updated their solution with a web based platform for answering as well (including apps for iOS and Android devices). This means that teachers still need a piece of software, but students can use a web based platform for responding. It supports multiple question types, but not math notation, as answers are only accepted as A,B,C,D etc. This does mean however, that nothing is hindering the teacher of asking a question in source code or math notation, and simply pointing out which is A and B. The iClicker software is capable of sending a screendump from the teachers computer to the web application (or app), this enabling the teacher to use any software of their choosing. In principle this enables multiline questions and math/source code notation.


\subsection{Mentimeter}
Mentimeter is a purely web based platform and also one of the more feature rich. It supports seven different question types, but does not have any multiline support. It does however offer a description field for each question, but it is limited to 100 characters making more complex question asking hard to do. Even though mentimeter is very feature rich, it does not have support for math notation, nor does it have source code notation.  



\begin{table}[]
\centering
\label{my-label}
\begin{tabular}{lllll}
 & Math  &  &  &  \\
 &  &  &  &  \\
 &  &  &  &  \\
 &  &  &  & 
\caption{My caption}
\end{tabular}
\end{table}







\subsection{Tophat}
Tophat is similar to Mentimeter and Poll Everywhere in form and function. It's a web based solution capable of taking attendance and asking questions. Tophat features 6 different question types, a discussion function and a tournament mode which let students compete agains each other.


\subsection{Learning Catalytics}
\subsection{Informaclicker}
\subsection{Renaissance}
%\subsection{Classroom Learning Partner}
%\subsection{Informa}






