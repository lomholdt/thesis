Question: What are the advantages and disadvantages of using a classroom response system in the classroom?

\subsection{Answers from Advanced Programming}\label{app:qualitative-advanced}
\begin{enumerate}
    \item Obvious

    \item It's good lol

    \item I don't see any disadvantages

    \item Increases my attention during the lectures

    \item good stuff

    \item It increases the attention of the class

    \item Fun and interactive 

    \item Makes the lecture more interesting

    \item Easy to check the understanding

    \item Student attention is higher. I see no immediate disadvantages, unless of course it is overused.

    \item If the questions are well designed, makes you think quickly and carefully

    \item More interaction between teachers and students

    \item Easier less personal interaction, but with awkward pauses.

    \item It keeps the students more focused on the class and its interactivity makes it fun to use.

    \item Only good things, anonymous

    \item Better response.

    \item It keeps you engaged and forces you to think. 

    \item Advantage: I can see if my class mates are on the same level.

    \item You are forced to think actively about what is taught during the lecture and apply it. Makes it easier to understand in the end

    \item I think it is very nice, you get to participate "anonymously" and get feedback on your understanding. TBH I did not know the first part of the questionnaire was related to the govote system at all.
advantage is that even shy people can be more involved.

    \item what system? 

    \item You are forced to think about an answer and actually give an answer.

    \item Just much more interaction, in the class, i Think it was really helpfull!

    \item Its easy to see if the class knows the curriculum in generel. Some people might be too shy to speak up during class even though they know the answers.

    \item Disadvange could be that people might not get the help they need during Classes.

    \item Measure your own progress and not just nodding along whatever the teacher says
\end{enumerate}


\subsection{Answers from Frameworks and Architectures for the Web}\label{app:qualitative-frameworks}

\begin{enumerate}

    \item It really helps examining yourself and seeing what you got or didn't get- at real time. I think it's obviously better if you can't see what others answered- or only at a later stage- so that you don't just get lazy and answer like everyone else before examining the question. :)
    
    \item It may be a slow way of going through material. The number of questions should probably be limited or done at different points during a lecture (e.g. after finishing up a specific topic before moving on to the next)
    
    \item advantages : interesting, can be more intensive disadvantages : less individual feedback
    
    \item  I think the system should be used without being able to look at the answers of the majority while answering the questions. Give people 30 seconds or 1 minute to answer before showing what people have answered.
    Students are easily biased, and it is hard to not look at what everyone else has answered before answering yourself.
    Advantage: The teacher is able to go through some materials that were not learned by the students.
    Disadvantage: If the the students are already active during the class, this tool kinda de-personalize the connection between the teacher and the students.
    Also: The students are only to give one answer, if the answer is wrong, the teacher might not know why the student answered this, and if the majority answered something different, a student may not say why they thought the wrong answer was correct... And then you'll have the: "Ok... I guess that's the right answer but I don't know why" effect again...

\end{enumerate}