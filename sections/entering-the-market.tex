\section{Entering the market of response systems}

%\todo{Vi skal ændre den her intro lidt. Vi kommer lige fra en lang beskrivelse af eksisterende løsninger.}

As shown in the previous section, there exists many different takes on CRSs, and such we wish to determine where our system will fit into the above. However, there a lot of different competitive areas which we have to navigate as well. To find this out we have used our own interpretation of \emph{Porters Five Forces} \cite{porter1979competitive}, which we will describe in the following section. 


\subsection{Porters Five Forces}
Figure \ref{fig:porter5forces} shows how competition can be seen as five forces. 
\emph{The threat of new entrants}, how new entrants to an industry have a desire to gain market share. \emph{The bargaining power of suppliers}, the dynamics of suppliers changing prices and capturing more value for themselves. \emph{The bargaining power of buyers}, the fact that buyers can demand more value by forcing down prices or wanting better quality. \emph{The threat of substitues}, other products on the market offering the same or better products by different means. For example using ITU's own system, LearnIT, with a native question feature instead of using a dedicated CRS. And finally, the fact that other systems compete for market share already. An example of strategies to achieve more market share could be by introducing new features or discounted pricing. Figure \ref{fig:porter5forces} shows a depiction of the model as it is explained in \citeA[p.~141]{porter1979competitive}.

% Threat of New Entry
% Supplier Power
% Threat of Substitution
% Buyer Power
% CENTER: Competitive Rivalry
% https://www.mindtools.com/pages/article/newTMC_08.htm

\begin{figure}[H]

\centering
\begin{tikzpicture}
[node distance = 1cm, auto,font=\footnotesize,
% STYLES
every node/.style={node distance=3cm},
% The comment style is used to describe the characteristics of each force
comment/.style={rectangle, inner sep= 5pt, text width=4cm, node distance=0.25cm, font=\scriptsize\sffamily},
% The force style is used to draw the forces' name
force/.style={rectangle, draw, fill=black!10, inner sep=5pt, text width=2.5cm, text badly centered, minimum height=1.2cm, font=\bfseries\footnotesize\sffamily}] 

% Draw forces
\node [force] (rivalry) {Rivalry among existing competitors};
\node [force, above of=rivalry] (substitutes) {Threat of substitutes};
%\node [force, text width=3cm, dashed, left=1cm of substitutes] (state) {Public policies};
\node [force, left=1cm of rivalry] (suppliers) {Bargaining power of suppliers};
\node [force, right=1cm of rivalry] (users) {Bargaining power of users};
\node [force, below of=rivalry] (entrants) {Threat of new entrants};

%%%%%%%%%%%%%%%
% Change data from here

% RIVALRY
\node [comment, below=0.25 of rivalry] (comment-rivalry) {
%(+) A war against Microsoft\\
%(+) Limiting sunk costs\\
%(+) Coopetition
};

% SUPPLIERS
\node [comment, below=0.25cm of suppliers] {
%(+) Efficiency\\
%(+) Attracting other developers\\
%(+) Creating a Chrome community
};

% SUBSTITUTES
\node [comment, right=0.25 of substitutes] {
%(+) Portability
};

% USERS
\node [comment, below=0.25 of users] {
%(+) Increasing the user information\\
%(+) Reducing the switching costs
};

% NEW ENTRANTS
\node [comment, right=0.25 of entrants] {
%(+) EC vs. Microsoft
};

% PUBLIC POLICIES
%\node [comment, text width=3cm, below=0.25 of state] {
%(+) Positively framed\\
%(+) Transparency\\
%(--) A new monopoly?
%};

%%%%%%%%%%%%%%%%

% Draw the links between forces
\path[->,thick] 
(substitutes) edge (rivalry)
(suppliers) edge (rivalry)
(users) edge (rivalry)
(entrants) edge (comment-rivalry);

\end{tikzpicture} 
\caption{Porters Five Forces }\label{fig:porter5forces}
\end{figure}


All forces must be taken into consideration when evaluating the strategic position to ensure our systems survival in a challenging market.
In the following section, we will walk through each of the five forces, describing their meaning and properties.
%In the following section we will determine how these forces can be interpreted against our system, and essentially against our strategy.

An important notice is, that there are plenty of fully digital CRSs available. The ones that we mention here is merely the ones that we were able to find during our research. Arguably there might me more, but we will only be concerned about the ones mentioned here. 

\subsubsection*{Threat of new entrants}\label{sec:threat-of-new-entrants}
The first force, the threat of new entrants, is primarily focused around the seven entry barriers, that incumbents have relative to new entrants, as explained by \citeA{porter2008five}.

The first barrier is \emph{supply-side economies of scale}. Essentially it covers the fact that the larger volumes, creates lower cost per unit. This is essentially true, due to the fact that a unit in our case can be seen as the users, and our cost per unit is server maintenance cost, the more users, the more servers but also spread fixed costs. For entrants though, the starting cost can be low and may simply scale as the business grows.

The \emph{demand-side benefits of scale}, is the fact that users might have increased willingness to buy a product if other buyers patronize the company. Users may also value being part of a bigger \emph{network}, thus this barrier is also known as the \emph{network effect} \cite[p.~81]{porter2008five}. It's hard to counter the fact that a huge volume will have a positive effect on almost any platform, but in general with CRS systems, it would seem plausible that systems are chosen based on user needs. This leads us naturally towards the third barrier \emph{customer switching costs}. The name is mostly self explanatory, but this barrier regards the switching cost to a new system. If we only consider web based systems, which most of the modern ones are, most users does not have any deep data affiliation with the systems, so switching comes at almost no cost, other than the possible hassle of getting used to something new. For entrants this is very positive. Given that gaining traction on such a market is very much possible if the system is feature competitive.

\emph{Capital requirements} consider the fact that the \emph{"need to invest large financial resources in order to compete can deter new entrants"} \cite[p.~81]{porter2008five}. In the case of CRS, it should be possible to build such a system without great capital requirements. In fact a working prototype can be made ready very quickly, as our example shows, though this is also a threat once you are in the market, since potential entrants can be plentiful.

\emph{Incumbency advantages independent of size} take into consideration the fact, that almost any incumbents has the advantage of already being available on the market \cite[p.~81]{porter2008five}. Simply put, being first can have potential advantages. Even though this might seem obvious, it is important to remember while analysing the CRS market. 

Entrants and incumbents might have \emph{unequal access to distribution channels}. For entrants, distribution channels must be secured in order to be able to displace others from the market. While dealing with non-physical products, and in this case dealing with software where the "shelf space" is unlimited, the distribution channels comes down to marketing of the product. Everybody has equal ability to market their product (online for example), but here the capital requirements might come in to play, since marketing can be a costly affair, and the distribution channels depend on it.

The final barrier is \emph{restrictive government policy}. It concerns the fact that the government might have direct influence on whether you will be able to even become an entrant on the market \cite[p.~82]{porter2008five}. Gathering potential licenses or other restrictions that might apply from a government level should be considered. While handling digital systems, laws governing privacy and data security should also be dealt with, though most CRS does not handle much user data beside an email or username and a password which is the case in our system.

\subsubsection*{Bargaining power of suppliers}
The next force is the \emph{bargaining power of suppliers}. This is concerned with the fact that powerful suppliers are able to manipulate different metrics to capture more value for themselves \cite[p.~82]{porter2008five}. This could be done by charging more for their product or limiting quality. \citeA{porter2008five} explains how Microsoft has contributed to the profitability of personal computers, and have been able to increase the price of their operating systems as an example of this force.

\subsubsection*{Bargaining power of buyers}
The next force is the \emph{bargaining power of buyers}. It is essentially the opposite of the \emph{bargaining power of suppliers} \cite[p.~83]{porter2008five}. Here buyers are able to force \emph{down} prices, demand \emph{better} quality etc. According to \citeA{porter2008five}, a customer group has negotiating power if they meet the listed criteria. For example, if there are only a few buyers or each buyer tends to buy in large volumes, as it is the case in the chemical industry \cite[p.~83]{porter2008five}. Also if a product is very standardized, buyers might find an equivalent product at another vendor, and if the switching cost is low, it might result in loosing buyers loyalty. 
On the other hand buyers are more price sensitive if for example the quality of the product is not affecting the buyers end product. Basically this means that if the product being bought is of high value for the customers outcome, the price tends to be of little concern \cite[p.~84]{porter2008five}. Also if the buyers earn low profits from the product, they are more attentive to prices. In contrast cash-rich buyers might care less. In relation, if the product has little effect on the buyers he might care less. \citeA{porter2008five}'s example of this, is within investment banking for example. You do not want to find the cheapest alternative when you want to invest money if it reflects the resulting outcome. In short \emph{"consumers tend to be more price sensitive if they are purchasing products that are undifferentiated, expensive relative to their incomes, and of a sort where product performance has limited consequences"} \cite[p.~84]{porter2008five}.

\subsubsection*{Threat of substitute products or services}
A substitute is a product, that serves the same or similar purpose as the original product, but by different means \cite[p.~84]{porter2008five}. Some of \citeA{porter2008five}'s own examples are videoconferencing that is a substitute for travel and e-mail that is a substitute for regular mail. A CRS could infact itself be a substitute, where hand raising and asking questions would be substituted by a CRS. 

% In our case a CRS is (or could be) a substitute for hand-raising.

The thread of substitution will be high if \emph{it offers an attractive price-performance trade-off to the industry's product} \cite[p.~84]{porter2008five}. In layman's terms the product should be competitive on price, compared to it's performance. Also if the buyers switching cost to a substitute product is low, the threat is high \cite[p.~84]{porter2008five}.

\subsubsection*{Rivalry among existing competitors}
The last and final force is the rivalry among existing competitors. Existing industry have to be competitive to survive, and doing discounts and adding new products to their product-portfolio is an important dimension \cite[p.~85]{porter2008five}. In general intense rivalry drives down an industry's potential profit based on number of competitors and their size, industry growth rates, exit barriers and rivals commitment \cite[p.~85]{porter2008five}. 

%\todo{Her slutter det meget brat, måske tilføje en lækker overgang eller noget}

In the sections above, we have shown the essence of Porters five forces. We have walked through each of the forces individually, and shown small examples for each, in order to be able to apply the model in the following sections.


















\subsection{Competitive analysis}







%Vi ser et problem ->
%Introduction
% Lit Review
% Method - hvordan går vi til det her problem?

%vi undersøger hvilke løsninger der er/markedet/konkurrence osv -> konklusion på det: Der er ingen der fylder hullet ud
% 'Competitor overview' - kunne ikke komme på et bedre navn - måske Existing solutions
% Competitive Analysis
% 

% threat of entry
Considering the complexity of most CRS'es, fortunately the economies of scale are relatively transparent. For a web application like this, the scaling can come, as more users start using it. For example, the hosting of the system could be done with as little as 3 servers, at a very low rate. If more capacity is needed, scaling horizontally could be as easy as the click of a button. The cost would of course then grow, but only when the user-base does as well.
The demand-side benefits of scale is, as explained in section \ref{sec:threat-of-new-entrants}, likely to be influenced by user needs. Many of the mentioned CRS are used \emph{'on the spot'}, and rarely hold features, that are critical for future use (such as grading). This is also backed by the fact, that the customer switching cost is so low on the web based systems. On the other hand, this also means, the possibility of users leaving our system without much consideration is relatively high, so gaining customer loyalty should be taken seriously.
Even though one could invest hours on end, designing and implementing and maintaining a CRS, it is indeed possible to build one without large financial resources. Of course, complexity can grow, depending on the system features, but taking the CRS definition into consideration, a \emph{plain} system adhering to this definition, should be very doable in a reasonable amount of time, without large financial support, even at a competitive level.

As mentioned, it is also a noticeable advantage to be first, and well established on a market. If the market is already saturated, entrants will have a really hard time entering the market, unless they have something extraordinary to offer. It would seem that many have tried (referring to the many systems mentioned here), and many also appears well established, but none seem to be dominant, and all offer a different set of features. This makes the idea of gaining tracking with niche features as a plausible strategy.

% power of suppliers
Even though this force is extremely relevant for the strategic planning of an entrant, the fact that our CRS is fully web based software, the supplier group that we are concerned about is less relevant. In our case we are primarily interested in hosting and data-storage, and the market for this is very versatile. One consideration might be that once a supplier has been chosen, the switching cost is high, due to the fact that it would be necessary to move all the data. Depending on the company's data history policy this could become the largest consideration in terms of bargaining power of suppliers, since the hosting providers could charge more, without overpricing the switching cost.


% power of buyers
The bargaining power of buyers is, however, more relevant when considering threads to ones position in the market of response systems. Since there's several classroom response systems on the market, a potential buyer may consider a lot of different systems before deciding on a system to use. The buyers of any CRS may consider their potential profits or any gain they will get before investing and implementing such a system into their courses and teaching. Since we aim at increasing interactivity in classrooms which teach IT courses, a potential buyer may ask what they will gain from using our system. When the buyer \emph{"... earns low profits, which creates great incentive to lower its purchasing costs."} \cite[p.~141]{porter1979competitive} price may be an important factor, because it can be hard to tell what the buyer will actually gain. Considering that our system is free of charge, we may be having a strong selling point before comparing features with direct competitors. Other free alternatives such as Kahoot and Socrative (see Table \ref{tab:overview-2}) offer similar features as we do, but they do however lack the ability to display code and mathematical expressions.


% threat of substitutes
It is necessary to take similar solutions and alternatives into account when considering threats of substitute products or services. At ITU there's a build in feature in LearnIT\footnote{A system based on Moodle which is an open-source learning platform. See \url{https://moodle.org}} which allows teachers to ask questions to students. The thread of substitution may be high if this feature's \emph{price-performance trade-off} is found acceptable by teachers and students. However, this feature is not as feature-rich as most CRS's and it may not fully replace any CRS. Since this feature is already present (at ITU) and paid for, the price-performance trade-off could be found to be acceptable and thus eliminate the need for an actual CRS. Other similar learning platforms may facilitate simple question features as well. It will therefore be necessary for us to offer more value than what a possibly already available system can provide, in order to be considered a better alternative or substitution. 


% Rivalry
As we have mentioned in the previous sections, there are quite a few CRSs on the market. They all vary in quality, features and pricing. This is an important aspect to consider when trying to gain market share. We have devoted section \ref{sec:competitor-overview} to show a list of relevant competitors and their features and pricing structures.








