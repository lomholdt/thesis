\section{Method}



The idea of Classroom Responses Systems is not a new one. As mentioned, early analog versions of these systems were used where students simply raised a card with their answer \cite[p.~257]{stowell2007benefits}. Several examples of such systems in more moderns versions have been shown, and such we must determine where our system lies from a strategic standpoint. To find this out we have used our own interpretation of \emph{Porters Five Forces} \cite{porter1979competitive}.

Furthermore, we have designed a test based on previous tests of response systems with the purpose of finding out if the systems is able to increase interaction in class. The test and the design of the test will be described in detail in this section.


\subsection{Porters Five Forces}
Figure \ref{fig:porter5forces} shows how competition can be seen as five forces. 
\emph{The threat of new entrants}, how new entrants to an industry have a desire to gain market share. \emph{The bargaining power of suppliers}, the dynamics of suppliers changing prices and capturing more value for themselves. \emph{The bargaining power of buyers}, the fact that buyers can demand more value by forcing down prices or wanting better quality. \emph{The threat of substitues}, other products on the market offering the same or better products by different means. For example using ITU's own system, LearnIT, with a native question feature instead of using a dedicated CRS. And finally, the fact that other systems compete for market share already. An example of strategies to achieve more market share could be by introducing new features or discounted pricing. Figure \ref{fig:porter5forces} shows a depiction of the model as it is explained in \citeA[p.~141]{porter1979competitive}.

% Threat of New Entry
% Supplier Power
% Threat of Substitution
% Buyer Power
% CENTER: Competitive Rivalry
% https://www.mindtools.com/pages/article/newTMC_08.htm

\begin{figure}[H]

\centering
\begin{tikzpicture}
[node distance = 1cm, auto,font=\footnotesize,
% STYLES
every node/.style={node distance=3cm},
% The comment style is used to describe the characteristics of each force
comment/.style={rectangle, inner sep= 5pt, text width=4cm, node distance=0.25cm, font=\scriptsize\sffamily},
% The force style is used to draw the forces' name
force/.style={rectangle, draw, fill=black!10, inner sep=5pt, text width=2.5cm, text badly centered, minimum height=1.2cm, font=\bfseries\footnotesize\sffamily}] 

% Draw forces
\node [force] (rivalry) {Rivalry among existing competitors};
\node [force, above of=rivalry] (substitutes) {Threat of substitutes};
%\node [force, text width=3cm, dashed, left=1cm of substitutes] (state) {Public policies};
\node [force, left=1cm of rivalry] (suppliers) {Bargaining power of suppliers};
\node [force, right=1cm of rivalry] (users) {Bargaining power of users};
\node [force, below of=rivalry] (entrants) {Threat of new entrants};

%%%%%%%%%%%%%%%
% Change data from here

% RIVALRY
\node [comment, below=0.25 of rivalry] (comment-rivalry) {
%(+) A war against Microsoft\\
%(+) Limiting sunk costs\\
%(+) Coopetition
};

% SUPPLIERS
\node [comment, below=0.25cm of suppliers] {
%(+) Efficiency\\
%(+) Attracting other developers\\
%(+) Creating a Chrome community
};

% SUBSTITUTES
\node [comment, right=0.25 of substitutes] {
%(+) Portability
};

% USERS
\node [comment, below=0.25 of users] {
%(+) Increasing the user information\\
%(+) Reducing the switching costs
};

% NEW ENTRANTS
\node [comment, right=0.25 of entrants] {
%(+) EC vs. Microsoft
};

% PUBLIC POLICIES
%\node [comment, text width=3cm, below=0.25 of state] {
%(+) Positively framed\\
%(+) Transparency\\
%(--) A new monopoly?
%};

%%%%%%%%%%%%%%%%

% Draw the links between forces
\path[->,thick] 
(substitutes) edge (rivalry)
(suppliers) edge (rivalry)
(users) edge (rivalry)
(entrants) edge (comment-rivalry);

\end{tikzpicture} 
\caption{Porters Five Forces }\label{fig:porter5forces}
\end{figure}


All forces must be taken into consideration when evaluating the strategic position to ensure our systems sustainability \todo{Find other word for this?} in a challenging market. 
In the following section, we will walk through each of the five forces, describing their meaning and properties.
%In the following section we will determine how these forces can be interpreted against our system, and essentially against our strategy.

An important notice is, that the amount of fully digital CRS systems are manyfold. The ones that we mention here is merely the ones that we were able to find during our research. Arguably there might me more, but we will only be concerned about the ones mentioned here. 


\subsubsection*{Threat of new entrants}\label{sec:threat-of-new-entrants}
The first force, the threat of new entrants, is primarily focused around the seven entry barriers, that incumbents have relative to new entrants, as explained by \citeA{porter2008five}.

The first barrier is \emph{supply-side economies of scale}. Essentially it covers the fact that the larger volumes, creates lower cost per unit. This is essentially true, due to the fact that a unit in our sense is users, and our cost per unit is server maintenance cost, the more users, the more servers but also spread fixed costs. For entrants though, the cost can be low and may simply scale as the business expands.

The \emph{demand-side benefits of scale}, is the fact that users might have increased willingness to buy a product if other buyers patronize the company. Users may also value being part of a bigger \emph{network}, thus this barrier is also known as the \emph{network effect} \cite[p.~81]{porter2008five}. It's hard to counter the fact that a huge volume will have a positive effect on almost any platform, but in general with CRS systems, it would seem plausible that systems are chosen based on user needs. This leads us naturally towards the third barrier \emph{customer switching costs}. The name is mostly self explanatory, but this barrier regards the switching cost to a new system. If we only consider web based systems, which most of the modern ones are, most users does not have any deep data affiliation with the systems, so switching comes at almost no cost. For entrants this is very positive. Given that gaining traction on such a market is very much possible if the system is feature competitive.

\emph{Capital requirements} consider the fact that the \emph{"need to invest large financial resources in order to compete can deter new entrants"} \cite[p.~81]{porter2008five}. In the case of CRS, it should be possible to build such a system without great capital requirements. In fact a working prototype can be made ready very quickly, as our example shows, thou this is also a threat once you are in the market, potential entrants can be plentiful.

\emph{Incumbency advantages independent of size} take into consideration the fact, that almost any incumbents has the advantage of already being available on the market \cite[p.~81]{porter2008five}. Simply put, being first can have potential advantages. Even though this might seem obvious, it is important to remember while analysing the CRS market. 

Entrants and incumbents might have \emph{unequal access to distribution channels}. For entrants, distribution channels must be secured in order to be able to displace others from the market. While dealing with non-physical products, and in this case dealing with software where the "shelf space" is unlimited, the distribution channels comes down to marketing of the product. Everybody has equal ability to market their product (online for example), but here the capital requirements might come in to play, since marketing can be a costly affair, and the distribution channels depend on it.

The final barrier is \emph{restrictive government policy}. It concerns the fact that the government might have direct influence on whether you will be able to even become an entrant on the market \cite[p.~82]{porter2008five}. Gathering potential licenses or other restrictions that might apply from a government level should be considered. While handling digital systems, laws governing privacy and data security should also be dealt with, though most CRS does not handle much user data beside an email or username and a password which is the case in our system.

\subsubsection*{Bargaining power of suppliers}
The next force is the \emph{bargaining power of suppliers}. This is concerned with the fact that powerful suppliers are able to manipulate different metrics to capture more value for themselves \cite[p.~82]{porter2008five}. This could be done by charging more for their product or limiting quality. \citeA{porter2008five} explains how Microsoft has contributed to the profitability of personal computers, and have been able to increase the price of their operating systems as an example of this force.

\subsubsection*{Bargaining power of buyers}
The next force is the \emph{bargaining power of buyers}. It is essentially the opposite of the \emph{bargaining power of suppliers} \cite[p.~83]{porter2008five}. Here buyers are able to force \emph{down} prices, demand \emph{better} quality etc. According to \citeA{porter2008five}, a customer group has negotiating power if they meet the listed criteria. For example, if there are only a few buyers or each buyer tends to buy in large volumes, as it is the case in the chemical industry \cite[p.~83]{porter2008five}. Also if a product is very standardized, buyers might find an equivalent product at another vendor, and if the switching cost is low, it might result in loosing buyers loyalty. 
On the other hand buyers are more price sensitive if for example the quality of the product is not affecting the buyers end product. Basically this means that if the product being bought is of high value for the customers outcome, the price tends to be of little concern \cite[p.~84]{porter2008five}. Also if the buyers earn low profits from the product, they are more attentive to prices. In contrast cash-rich buyers might care less. In relation, if the product has little effect on the buyers he might care less. \citeA{porter2008five}'s example of this, is within investment banking for example. You do not want to find the cheapest alternative when you want to invest money if it reflects the resulting outcome. In short \emph{"consumers tend to be more price sensitive if they are purchasing products that are undifferentiated, expensive relative to their incomes, and of a sort where product performance has limited consequences"} \cite[p.~84]{porter2008five}.

\subsubsection*{Threat of substitute products or services}
A substitute is a product, that serves the same or similar purpose as the original product, but by different means \cite[p.~84]{porter2008five}. Some of \citeA{porter2008five}'s own examples are videoconferencing that is a substitute for travel and e-mail that is a substitute for regular mail. 

% In our case a CRS is (or could be) a substitute for hand-raising.

The thread of substitution will be high if \emph{it offers an attractive price-performance trade-off to the industry's product} \cite[p.~84]{porter2008five}. In layman's terms the product should be competitive on price, compared to it's performance. Also if the buyers switching cost to a substitute product is low, the threat is high \cite[p.~84]{porter2008five}.

\subsubsection*{Rivalry among existing competitors}
The last and final force is the rivalry among existing competitors. Existing industry have to be competitive to survive, and doing discounts and adding new products to their product-portfolio is an important dimension \cite[p.~85]{porter2008five}. In general intense rivalry drives down an industry's potential profit based on number of competitors and their size, industry growth rates, exit barriers and rivals commitment \cite[p.~85]{porter2008five}. 

\todo{Her slutter det meget brat, måske tilføje en lækker overgang eller noget}

In the sections above, we have shown the essence of Porters five forces, and what the model looks like. We have walked through each of the forces individually, and shown small examples for each, in order to be able to apply the model in the following sections.


































\subsection{Testing CRSFIT in a classroom}\label{sec:testingcrs}
We have designed a test in order to tell whether or not our solution can have an impact on teaching classes that covers teaching programming and mathematics. The test is to some extend inspired by \cite{siau2006use} and their work. \citeA{siau2006use} found that there wasn't developed any instrument for measuring interactivity. This resulted in \citeA{siau2006use} to create their own method, from which we have taken elements to support our own.

The test by \citeA{siau2006use} is designed to tell whether or not they are able to move peoples behaviour, beliefs and attitude towards engagement, interactivity and participation during class. The test they designed consists of a \emph{pretest}, \emph{implementation} (of the system in a class) and a \emph{posttest} followed up by a \emph{qualitative data collection}. 

Our test is structured in the same way and we are asking similar questions in the different parts of the test as well. The whole test should preferably be carried out over a whole semester, but due to the limited time of this project the test will be carried out in a single lecture. The pretest and posttest are made as questionnaires. 


\subsubsection*{Measuring interactivity}
By having a pretest and a posttest asking the same questions \citeA{siau2006use} was able to compare the results and measure whether or not they were able to move peoples attitude. Their pretest consisted of two parts which were asking about individual interactivity and general interactivity in the class. The questions in both parts of the pretest were formulated as statements such as \emph{"I am engaged in class."} and \emph{"I provide my opinion to questions from the teacher during the class."}. This part of their test will be almost identical to our test. Our test differ from the theirs in the posttest. Since we are not able to carry out our test over a longer period of time, our posttest will not be asking into \emph{individual interactivity} and \emph{general interactivity} again. We will, however, use the results regarding \emph{individual interactivity} and \emph{general interactivity} from the pretest to tell to which degree students interact in class in general. The pretest should be answered by students before or in the beginning of the given lecture in which the test is carried out (See appendix \ref{app:pretest} for a complete list of questions from the pretest questionnaire). 

During the lecture we will ask students questions relevant to their class with our solution, CRSFIT. This part of the test is the \emph{implementation}. Students will answer the questions anonymously and they will not be held accountable for their answers. The purpose of this part of the test is to 1) tell if students actually benefit from using the system and overall find it useful, and 2) test our solution in a real setting.

It requires training and experience to get the most out of a classroom response system. The literature and insights about how to use response systems in a beneficial way has developed alongside with the research of the implementation of response system in classrooms. For example, see: \cite{lantz2014effectiveness,draper2004increasing,lin2011implementing} for insights and comments about how to effectively use response systems. We will develop questions, in collaboration with our supervisor, Peter Eklund, which will be asked during the lecture in which we carry out our test.
The questions that where asked during the test can be found in appendix \ref{app:questions} and on the live site at \url{crsfit.online/r/b4tc8}.


\subsubsection*{Acceptance of technology}
In the posttest questionnaire we will ask students about the perceived ease of use of CRSFIT and the perceived usefulness. The \emph{perceived ease of use} is included in order to get feedback on how easy the system is to use. This could give us a clear impression since no participants have seen or used the system prior to the test. The \emph{perceived usefulness} asks into the idea and possible benefits of the system with statements like \emph{"Using the system makes it easier for me to interact in the class."} to be rated. Perceived ease of use and perceived usefulness is adopted from the \emph{Technology Acceptance Model (TAM)} \cite{siau2006use,davis1989user} where they are defined as follows:

\emph{"Perceived usefulness (U) is defined as the prospective user's subjective probability that using a specific application system will increase his or her job performance within an organizational context. Perceived ease of use (EOU) refers to the degree to which the prospective user expects the target system to be free of effort."} \cite[p.~985]{davis1989user}

The model originally aims to predict users intention to adopt new technology. We will, however, not use the model as intended. Inspired by the questions in a different research by \citeA{davis1989perceived} we will ask similar styled questions in our posttest though.

This part of the test will tell if the students feel an individual increased interactivity by using our system. Also, we ask the students whether or not they use any tools to test their answers which includes any code/script or mathematical expressions. Included in the posttest questionnaire is also an open ended question in which we ask students to give their opinion about advantages and disadvantages of using a classroom response system. This is the qualitative part of the test. This part serves to give us an idea about if there are areas that this kind of system does particular good or bad. See appendix \ref{app:posttest} for a complete list of the questions from the posttest questionnaire. 


\subsubsection*{The pretest and posttest questionnaires} % Field testing the system with quantitative and some qualitative data
% Likert scale
% Spørgsmål - se bog. Structured format (i form af at spørgsmålene ikke bliver spurgt af os, men står på tekst (spørgeskema). Likert skala, 1-9. (Find artikel)
% Open ended spørgsmål til at slutte af med - kvalitativ data som vi vil bruge til ?? i analysen.
The pretest and the posttest are designed as questionnaires. Questionnaires makes the least demands on personal and social skills of the questioner \cite[p.~74]{deacon2007researching} and enables us to measure the results. The questionnaires are to be answered online with Google Forms. Our way of asking the questions are highly structured since they are written down and asked in the exact same way to all respondents \cite[p.~65]{deacon2007researching}. Both questionnaires can be found in the appendix (\ref{app:pretest}, \ref{app:posttest}).

When asking questions in a questionnaire (or in general) the questioner should be aware which type of questions are asked. What does the question seek to find information about? \citeA[pp.~80-84]{dillman1978mail} lists four types of questions which are asking into either \emph{behaviour}, \emph{beliefs}, \emph{attitudes} or \emph{attributes}. With our questions we are interested in behaviour, beliefs and attitudes. We are not focusing on attributes. Questions asking into attributes are concerned about background information of the respondent such as age, gender etc \cite[p.~75]{deacon2007researching}. While these questions typically are easy to answer, they are of no interest to us. They are often used to say something about a group of people you don't know and are of most use in very large samples. Our questions are interested in behaviour when we ask student to rate statements such as \emph{"I interact with the teacher in class."} and beliefs (\emph{"Students receive feedback from the teacher during the class."}) and attitudes (\emph{"I find the system useful in enhancing my interaction in the class."}).

It may seem that some of our questions are identical to each other even though they are slightly different. Because of this we risk that respondents doesn't see the difference from the previous question and therefore may answer the same as in the previous question. Because of the similarity we could risk that respondents find the questionnaire as a whole to be frustrating or difficult to answer.

All of the questions in our questionnaires (except two) are formulated as statements which should be answered on a Likert scale \cite{likert1932technique} with items 1-9 where 1 is \emph{Strongly disagree} and 9 is \emph{Strongly agree}. With an uneven number we offer a neutral choice (5 on our scale) in order to avoid forcing an opinion on the respondents. We could have chosen to make the scale smaller or larger. A smaller scale than 1-9 could result in less different answers since the options are fewer. \citeA[p.~507]{matell1972there} found that less than 7 options on a Likert scale resulted in greater usage of the "neutral" choice. In order to avoid that we choose 9 options which we hope will not offer too many options as well.

The two questions not concerning 

%To sum things up the following list shows what we'll do and when:

%\begin{itemize}
%    \item Pretest: Questionnaire, before or in the beginning of the lecture
%    \item Implementation: Students answer questions with CRSFIT during lecture
%    \item Posttest + qualitative data: Questionnaire, in the end of the lecture
%\end{itemize}
















% T-test i stedet for chi-square fordi T-test ikke kræver 5 svar i hver celle for at kunne bruges.

% T-test for 'dependent samples'/'correlated samples' hvor man "eksperimenterer" og sammenligner før og efter - præcis som vi har tænkt os at gøre.

% Side 110!!!

% Null hyptohesis: "Der er ingen forskel mellem ... "

% Vi har et two-way relationship mellem de spørgsmål der er identiske (cross-tabulation og correlation)